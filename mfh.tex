%% Modified version of ModernCV Latex template for my CV
%% Muh. Fuad Hasan (March 2013)
%% Copyright 2006-2010 Xavier Danaux (xdanaux@gmail.com)
%  TODO:
% - Keep to update with more entries
% - Clean-up, make simple, brief but informative
% - proofreading (check grammar and spelling)

%----------------------------------------------------------------------------------
%            Pre (done)
%----------------------------------------------------------------------------------
\documentclass[10pt, a4paper]{moderncv}

\moderncvtheme[blue, roman]{classic}                    % default theme classic

% only for xelatex
\usepackage{ifxetex}
\ifxetex                								
    \usepackage{xltxtra}                                % this should load both fontspec & xunicode
    \usepackage{fontspec}
    \setromanfont{Helvetica Neue}
\else
    \usepackage[utf8]{inputenc}
    \usepackage[T1]{fontenc}
\fi

% adjust the page margins
\usepackage[scale=0.8]{geometry}
\setlength{\hintscolumnwidth}{3cm}                      % if you want to change the width of the column with the dates
\AtBeginDocument{\setlength{\maketitlenamewidth}{6cm}}  % if you want to change the width of your name placeholder (to leave more space for address details)
\AtBeginDocument{\recomputelengths}                     % required when changes are made to page layout lengths

% for publications section
\renewcommand{\refname}{Publications}

% personal info header
\firstname{}
\familyname{}
\photo{picture}
\title{Curriculum Vitae} 
\address{}{Jakarta, Indonesia}
\mobile{+62 857 789 897 17}
\email{fuad\_hasan@fastmail.jp}
\homepage{www.linkedin.com/in/mfuadhasan}

%\fontspec[Ligatures={Common, Historical}]{Helvetica Neue}
%\fontsize{10pt}{18pt}\selectfont @ngoles

% to show numerical labels in the bibliography; it useful if need to put citations
\makeatletter
\renewcommand*{\bibliographyitemlabel}{\@biblabel{\arabic{enumiv}}}
\makeatother

%\newcommand\Colorhref[3][cyan]{\href{#2}{\small\color{#1}#3}}
\usepackage[usenames,dvipsnames]{xcolor}
\definecolor{darkgray}{rgb}{0.4,0.4,0.4}
\newcommand\Colorhref[1]{\small\textcolor{RoyalBlue}{\url{#1}}}


%----------------------------------------------------------------------------------
%            Intro (done)
%----------------------------------------------------------------------------------
\begin{document}
\maketitle

\begin{small}
\textcolor{darkgray}{
\textbf{I am a strongly motivated individual, working to improve my knowledge, skills, and abilities
on a daily basis. Always interested in exciting and challenging projects in creative environments
where I can both contribute and learn from a great team of people.}}
\end{small}


%----------------------------------------------------------------------------------
%            Personal (done)
%----------------------------------------------------------------------------------
\section{Personal}
\cvline{Full Name}{Muhammad Fuad Hasan}
\cvline{Place, Date of Birth}{Ujung Pandang, Indonesia, July 6 1986}
\cvline{Sex, Marital Status}{Male, Single}
%\cvline{Religion}{Islam}
%\cvline{Home Address}{Jl. Mampang Prapatan XI No.70 RT.1 RW.4, Jakarta Selatan, Indonesia 12790} 


%----------------------------------------------------------------------------------
%            Education (done)
%----------------------------------------------------------------------------------
\section{Education}
\cventry{\begin{small}Oct 2011\end{small}}{Bachelor of Engineering in Informatics Engineering}{STT Telkom (IT Telkom)}{Bandung, Indonesia}{}
{
    \begin{scriptsize}
    Final Project: "Signaling Performance Analysis of Mobile IPV6 Integrated on MPLS Network with XCAST6".\\
    \end{scriptsize}
}


%----------------------------------------------------------------------------------
%            Experience (more explanation: done)
%----------------------------------------------------------------------------------
\section{Experience}
\cventry{\begin{small}Nov 2012--Current\end{small}}{Software Developer (Freelance)}{Kickstart Interactive (startup)}{}{Jakarta, Indonesia} 
{ 
    \begin{scriptsize} 
    Contribute to the development of "Paramanusa Filing" application product with target market
    for small and middle enterprise.\\
    \end{scriptsize}
}
\cventry{\begin{small}Nov 2011--Oct 2012\end{small}}{Network Engineer}{PT. Bigjava (bigjava.com)}{}{Jakarta Selatan, Indonesia} 
{ 
    \begin{itemize} 
    \item Develop and deploy network application product with target market for telecommunication companies in Indonesia.
    \item Meet with clients to establish software specifications and also doing product presentation to potential customers.
    \item Lead a small team of 2-3 engineers to work on a project in specific product development.\\
    \end{itemize}
}
\cventry{\begin{small}Jul--Sep 2009\end{small}}{Network Engineer (Intern)}{PT. Bandung Sinergi Akses Teknologi (Greenlinks ISP)}{}{Bandung, Indonesia}
{
    \begin{scriptsize}
    Monitor and study network performance of Greenlinks data center as information
    basis in planning to improve its network service to end-clients.\\
    \end{scriptsize}
}
\cventry{\begin{small}Jul--Oct 2008\end{small}}{Network Technician (Freelance)}{Independent Contractor}{}{Bandung, Indonesia}
{
    \begin{scriptsize}
    Install, configure, and maintain computer network infrastructure for blocks home residence.\\
    \end{scriptsize}
}
\cventry{\begin{small}Jul--Aug 2007\end{small}}{Customer Complaint Service Officer (Intern)}{PT. Telekomunikasi Indonesia}{}{Bandung, Indonesia}
{
    \begin{scriptsize}
    Handle formal complaint from customer and develop a web application to help managing
    complaint's problem solving process.\\
    \end{scriptsize}
}
\\

%----------------------------------------------------------------------------------
%            Skills (more explanation, fix this first)
%----------------------------------------------------------------------------------
\section{Skills}
\cvline{Languages}{C++, Python, Perl, Shell Script, Java, PHP, SQL, \LaTeX, Scheme, Go}     % computer language
\cvline{Protocols \& APIs}{ XML, JSON, SOAP, REST}                                          %
\cvline{Software}{gcc, gdb, BASH/ZSH, Vim, Microsoft office, XFCE, KDE, GNOME etec}         % software, need framework qt mpi etc
\cvline{Database}{MongoDB, MySQL, MSSQL, Redis, and PostgreSQL}                             % database
\cvline{Hardware}{ATCA platform, IBM, HP, and Dell Blade Server}                            % hardware
\cvline{Operating System}{GNU/Linux (RedHat/CentOS, SuSE, Slackware, and Debian/Ubuntu), BSD (FreeBSD,
    OpenBSD, NetBSD, OpenDarwin, and OpenSolaris), Apple OS X, Microsoft
    Windows (desktop/server)}                                                               % operatingsystems
\cvline{Virtualization}{VMware, VirtualBox, Qemu}                                           % virtual
\cvline{SCM}{Git, Mercurial, and CVS}                                                       % cvs
\cvline{Log Management}{Splunk, Logstash}                                                   % 
\cvline{Network Simulator}{Omnet++, NS2, Opnet, GNS3}                                       % simulator, need numerial comp
\cvline{Network Monitoring}{Nagios, OpenNMS}                                                % network monitoring
\cvline{Network Service}{DNS (Bind), web proxy (Squid), web server (Apache,
    Lighttpd, Nginx), ftp server (vsftpd), mail server (qmail, postfix,
    sendmail), vpn server (OpenVPN, mpd), VoIP server (Asterisk and SER)}                   % network serv, need firewall ids etc
\cvline{Packet Analyzer}{Wireshark, TCPdump/snoop}                                          % packet analyzer 
\cvline{Idioms}{Indonesian (native proficiency), English (professional working proficency)} % human language 
% maybe i can add some for non-categorize points like quick learning/adapt, network protocol knowledge etc


%----------------------------------------------------------------------------------
%            Projects & Interests (i'll put this on if it really needed)
%----------------------------------------------------------------------------------
%\section{Projects \& Interests}
%\cventry{\begin{small}July 2009--Current\end{small}}{Stack Overflow}{}{}{}
%{
%    \begin{small}Average top 15\% user, with a ~2.5k reputation.\end{small}\\
%    \Colorhref{http://stackoverflow.com/users/145077/mr-gando}\\
%}
%\cventry{\begin{small}May 2011--Current\end{small}}{Github}{Xcode 4 Template Generator}{}{}
%{
%    \begin{small}Python script to help in the process of making Xcode 4 Templates.\end{small}\\
%    \Colorhref{https://github.com/MrGando/Xcode-4-Template-Generator}\\
%}
%\cventry{\begin{small}15+ years\end{small}}{Piano Playing}{}{}{}
%{
%    Have been studying piano nearly all my life, music is a huge part of me.
%}


%----------------------------------------------------------------------------------
%            Activities (6 entries: done)
%----------------------------------------------------------------------------------
\section{Activities}
\cventry{\begin{small}Nov 2010--Mar 2011\end{small}}{Member}{Study Group of eLearning Lab.}{IT Telkom Bandung, Indonesia}{}
{
    \begin{scriptsize}
    Study on eLearning technology particularly web based application eLearning
    and IEEE Learning Technology System Architecture (LTSA) framework which
    operate in an environment of distributed system.\\
    \end{scriptsize}
}
\cventry{\begin{small}Dec 2009--Oct 2010\end{small}}{Member}{Study Group of Data Mining Lab.}{IT Telkom Bandung, Indonesia}{}
{
    \begin{scriptsize}
    Study and research for implementing K-nearest Neighbor Algorithm and Hidden
    Markov Model for machine learning application.\\ 
    \end{scriptsize}
}
\cventry{\begin{small}Sep 2007--Aug 2009\end{small}}{Researcher \& Teaching Assistant}{Teleinformatics \& Operating System Lab.}{IT Telkom Bandung, Indonesia}{}
{
    \begin{itemize}
    \item Doing practical research: "Implementing Failover Firewalls System using CARP Protocol on OpenBSD". 
    \item Teaching basic computer network course material for students in class and laboratory practice.\\
    \end{itemize}
}
\cventry{\begin{small}Aug 2006--May 2007\end{small}}{Activist}{Linux User Group (LUG)}{STT Telkom Bandung, Indonesia}{}
{
    \begin{scriptsize}
    Participate actively to help promoting and teaching open source software in my local campus community.\\
    \end{scriptsize}
}
\cventry{\begin{small}Dec 2004--Oct 2006\end{small}}{Journalist}{Masyarakat Jurnalistik}{STT Telkom Bandung, Indonesia}{}
{
    \begin{scriptsize}
    Learned about journalistic and wrote news articles for my campus journal/magazine as amateur journalist.\\
    \end{scriptsize}
}
\cventry{\begin{small}Nov 2004--May 2005\end{small}}{Member}{Study Group of Artificial Intelligence Lab.}{STT Telkom Bandung, Indonesia}{}
{
    \begin{scriptsize}
    Study fundamental theory about Artificial Intelligence and its simple algorithms implementation in software.\\
    \end{scriptsize}
}


%----------------------------------------------------------------------------------
%            Short Course & Seminar (11 entries: done)
%----------------------------------------------------------------------------------
\section{Short Course \& Seminar}
\cventry{\begin{small}Jun 2012\end{small}}{Cisco Innovate 2012}{Seminar}{by Cisco Systems}{Jakarta, Indonesia}
{
    \begin{scriptsize}
    As representative for my employer (PT. Bigjava) to learn Cisco's new
    innovations that can be used as technical references to improve my company's
    technology products.\\
    \end{scriptsize}
}
\cventry{\begin{small}Feb 2011\end{small}}{Qt Workshop for Developer}{Short Course}{by Nokia Indonesia}{Bandung, Indonesia}
{
    \begin{scriptsize}
    As software developer, I came to learn developing application using Qt framework
    as infrastructure basis that would be run on Nokia's MeeGo mobile OS platform.\\
    \end{scriptsize}
}
\cventry{\begin{small}Feb 2010\end{small}}{Security in eBanking}{Seminar}{by Switching Technique Lab. IT Telkom}{Bandung, Indonesia}
{
    \begin{scriptsize}
    As seminar participant, I came to learn technology security issues and trend in financial
    transaction via online eBanking especially at banks in Indonesia.\\
    \end{scriptsize}
}
\cventry{\begin{small}Feb 2009\end{small}}{IP Multimedia Sub System (IMS) Technology}{Seminar}{by Switching Technique Lab. IT Telkom}{Bandung, Indonesia}
{
    \begin{scriptsize}
    As seminar participant, I came to learn about IMS technology implementation
    in telecommunication industry and its open source implementation like OpenIMS that
    can be used to build core element of IMS architecture.\\
    \end{scriptsize}
}
\cventry{\begin{small}Sep 2008\end{small}}{Ruby on Rails Application Development}{Short Course}{by Software Engineering Lab. IT Telkom}{Bandung, Indonesia}
{
    \begin{scriptsize}
    As student, I learned to build application using Ruby on Rails framework for
    rapid web application prototyping.\\
    \end{scriptsize}
}
\cventry{\begin{small}Sep 2007\end{small}}{WiMax Technology}{Short Course}{by Mobile Communication Lab. STT Telkom}{Bandung, Indonesia}
{
    \begin{scriptsize}
    As student, I learned basic WiMax technology, deployment, and its application in
    telecommunication industry.\\
    \end{scriptsize}
}
\cventry{\begin{small}Sep 2006\end{small}}{eBusiness and Portal Service}{Seminar}{by Software Engineering Lab. STT Telkom}{Bandung, Indonesia}
{
    \begin{scriptsize}
    As seminar participant, I learned about economic potential of online
    business based on portal service web application which can be built using
    web frameworks.\\
    \end{scriptsize}
}
\cventry{\begin{small}May 2006\end{small}}{Softswitch Training Using Asterisk}{Short Course}{by Telecommunication Access Lab. STT Telkom}{Bandung, Indonesia}
{
    \begin{scriptsize}
    As student, I learned to setup and configure Asterisk softswitch for campus telephone/VoIP exchange.\\
    \end{scriptsize}
}
\cventry{\begin{small}Mar 2005\end{small}}{Open Source and Network Security}{Seminar}{by Teleinformatics Lab. STT Telkom}{Bandung, Indonesia}
{
    \begin{scriptsize}
    As seminar participant, I learned about open source software that can be used
    as tool for doing network security assessment and hardening computer
    networks.\\
    \end{scriptsize}
}
\cventry{\begin{small}Jan 2005\end{small}}{Relational Database Fundamental Training}{Short Course}{by Database Lab. STT Telkom}{Bandung, Indonesia}
{
    \begin{scriptsize}
    As student, I learned to create relational database application using ER diagram as
    a design basis and implementing it on MS SQL platform.\\
    \end{scriptsize}
}
\cventry{\begin{small}Dec 2004\end{small}}{Workshop Computer Hardware \& Troubleshooting}{Short Course}{by Computer Hardware Lab. STT Telkom}{Bandung, Indonesia}
{
    \begin{scriptsize}
    As student, I learned to assemble personal computer (PC) hardware and how to
    do computer troubleshooting.\\
    \end{scriptsize}
}


%----------------------------------------------------------------------------------
%            References
%----------------------------------------------------------------------------------
\section{References}
\cventry{}{Available upon Request}{}{}{}
{
    %list of people as references
}

\end{document}

